\documentclass[11pt,a4paper]{article}
\usepackage[utf8]{inputenc}
\usepackage{float,graphicx,amsfonts,amsmath,amssymb,authblk,graphicx,longtable,booktabs,fullpage}
\usepackage{wrapfig}
\graphicspath{ {../img/} }
\usepackage[colorlinks = true,
            linkcolor = black,
            urlcolor  = blue,
            citecolor = blue,
            anchorcolor = blue]{hyperref}


\title{%
  Misura della distanza focale di una lente convergente \\
  \large Relazione: Esame di Laboratorio di Fisica 2}
    

\author[1]{Dennis Angemi}%
\affil[1]{Dipartimento di Fisica e Astronomia ``Ettore Majorana'', Università degli Studi di Catania}%
\date{dd giugno 2023}

\begin{document}

\maketitle

\section{Introduzione}
\subsection{Scopo della misura}
\subsection{Cenni teorici}

\section{Descrizione dell’ apparato sperimentale}
\subsection{Principio di funzionamento degli apparecchi}
\subsection{Caratteristiche degli strumenti di misura}

\section{Esecuzione dell’ esperienza}
\subsection{Procedura}
\subsection{Dati sperimentali}

\begin{longtable}[]{@{}lll@{}}
    \toprule
    configuration & $x_p \; [cm]$ & $x_o \; [cm]$ \tabularnewline
    \midrule
    \endhead
    1 & 4.60 $\pm \; 0.05$ & 10.10 $\pm \; 0.05$ \tabularnewline
    \bottomrule
    \label{tab:conf6}
    \\
    \caption{Dati grezzi configurazione 1}
\end{longtable}
    
dove
\begin{itemize}
    \item $x_p$ rappresenta la posizione del proiettore;
    \item $x_o$ rappresenta la posizione dell'oggetto;
\end{itemize}

\begin{longtable}[]{@{}llllll@{}}
    \toprule
    measure ID & $x_l$ & $x_{s,inf}$ & $x_{s,sup}$ & $l_{inf}$ & $l_{sup}$ \tabularnewline
    $.$ & $\pm \; 0.05 \; cm$ & $\pm \; 0.05 \; cm$ & $\pm \; 0.05 \; cm$ & $\pm \; 0.005 \; cm$ & $\pm \; 0.005 \; cm$ \tabularnewline
    \midrule
    \endhead
    C1M1 & 21.80 & 66.30 & 78.40 & 6.360 & 7.850 \tabularnewline
    C1M2 & 22.50 & 60.40 & 65.90 & 5.090 & 6.230 \tabularnewline
    C1M3 & 23.00 & 59.90 & 64.90 & 5.065 & 6.095 \tabularnewline
    C1M4 & 23.50 & 56.85 & 58.90 & 4.235 & 4.850 \tabularnewline
    C1M5 & 24.00 & 55.30 & 56.80 & 3.985 & 4.320 \tabularnewline
    C1M6 & 24.50 & 53.70 & 54.30 & 3.600 & 3.275 \tabularnewline
    C1M7 & 24.80 & 52.85 & 54.50 & 3.265 & 3.890 \tabularnewline
    C1M8 & 25.00 & 52.80 & 53.80 & 3.410 & 3.535 \tabularnewline
    C1M9 & 25.30 & 51.90 & 53.00 & 3.150 & 3.995 \tabularnewline
    C1M10 & 25.50 & 51.60 & 52.80 & 3.000 & 3.285 \tabularnewline
    \bottomrule
    \label{tab:conf6}
    \\
    \caption{Dati grezzi configurazione 1}
\end{longtable}

dove
\begin{itemize}
    \item $x_l$ rappresenta la posizione della lente;
    \item $x_{s,inf}$ e $x_{s,inf}$ rappresentano rispettivamente la posizione dell'estremo inferiore de di quello superiore dell'intervallo in cui l'immagine appare nitida sullo schermo;
    \item $l_{inf}$ e $l_{sup}$ rappresentano rispettivamente l'estremo inferiore e superiore dell'intervallo delle lunghezze dell'immagine della fenditura per cui essa fosse nitida;
\end{itemize}

\section{Analisi dei dati}

\section{Conclusione}

\section{Note aggiuntive}

\subsection{Data availability}
The data that support the findings of this study are openly available in \href{https://github.com/dennisangemi/lab2-exam/tree/main/data}{dennisangemi/lab2-exam GitHub Repository} at \href{https://github.com/dennisangemi/lab2-exam}{https://github.com/dennisangemi/lab2-exam/tree/main/data} under \href{https://creativecommons.org/licenses/by/4.0/}{CC-BY 4.0 license}.

\subsection{Code availability}
The MATLAB code written to get the findings of this study is openly available in \href{https://github.com/dennisangemi/lab1-exam/tree/main/scripts}{dennisangemi/lab1-exam GitHub Repository} at \href{https://github.com/dennisangemi/lab1-exam/tree/main/scripts}{https://github.com/dennisangemi/lab1-exam/tree/main/scripts}

\subsection{Software usati}
\begin{itemize}
    \item \textbf{Google Sheets}: Data Collection
    \item \textbf{MATLAB}: Data Analysis
    \item \textbf{GitHub}: Resource sharing
    \item \textbf{Figma}: Images designing
\end{itemize}

\section{Bibliography}
\begin{itemize}
    \item Taylor,~ J. (1999).~\emph{Introduzione all'analisi degli errori: Lo
  studio delle incertezze nelle misure fisiche.~}Zanichelli
    \item Bevington, P. (2002).~\emph{Data Reduction and Error Analysis for the Physical Sciences.~} McGraw-Hill Education ~
    \item Malthe-Sørenssen, A. (2015). \emph{Elementary Mechanics Using Matlab: A Modern Course Combining Analytical and Numerical Techniques}. Springer
    \item Mazzoldi, P., Nigro, M., Voci, C. (2001). \emph{Fisica. Meccanica, termodinamica (Vol. 1)}. Edises
\end{itemize}



\subsection{Bibliografia/sitografia}
\begin{itemize}
    \item scheda esperienza https://cms333.ct.infn.it/~costa/Lab2/Schede/Lente-Convergente/Lente-Convergente.pdf
\end{itemize}

\end{document}

